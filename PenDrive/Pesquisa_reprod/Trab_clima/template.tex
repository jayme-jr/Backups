\documentclass[a4paper]{article}
\usepackage[utf8]{inputenc}
\usepackage[brazil]{babel}

\usepackage{geometry}
\geometry{a4paper, left=40mm, right=40mm, top=25mm, bottom=25mm}

% Cores: http://latexcolor.com/.
\usepackage{xcolor}
\definecolor{utahcrimson}{rgb}{0.83, 0.0, 0.25}
\newcommand{\tc}[1]{\textcolor{utahcrimson}{#1}}

\usepackage{hyperref}
\hypersetup{colorlinks = true}

\title{Análise do Clima em Curitiba para o período AAA a BBB}
\author{Acadêmico HHH}
\date{20 de março de 2018}

\usepackage{Sweave}
\begin{document}
\Sconcordance{concordance:template.tex:template.Rnw:%
1 19 1 1 0 7 1 1 2 1 0 1 5 7 0 1 2 2 1 1 4 3 0 1 2 1 1 5 0 1 3 6 0 1 1 %
5 0 1 1 5 0 1 1 5 0 1 1 5 0 1 1 5 0 1 1 5 0 1 1 5 0 1 1 5 0 1 1 5 0 1 1 %
5 0 1 1 5 0 1 1 5 0 1 2 1 1 1 2 1 14 16 0 1 2 3 1}


\maketitle

Use a construção \verb|\tc{|\verb|\|\verb|Sexpr{<cmd>}}| para destacar o uso dos
recursos do Sweave/Knitr.

\begin{Schunk}
\begin{Sinput}
> so <- Sys.info()["sysname"]
> if (so == "Linux") {
+     frase <- "A propósito, parabéns por usar Linux"
+ } else {
+     frase <- sprintf("Como assim você usa %s? Use Linux", so)
+ }
\end{Sinput}
\end{Schunk}

Esse relatório foi feito no \tc{Linux}. \tc{A propósito, parabéns por usar Linux}.

\begin{Schunk}
\begin{Sinput}
> # Sobre detalhes de como formatar datas.
> 
> x <- "20/03/2018 22:00:00"
> dt <- as.POSIXct(x = x, format = "%d/%m/%Y %H:%M:%S")
> dt
\end{Sinput}
\begin{Soutput}
[1] "2018-03-20 22:00:00 BRT"
\end{Soutput}
\begin{Sinput}
> # Extração de partes da estampa.
> strftime(dt, format = "%A") # Dia da semana.
\end{Sinput}
\begin{Soutput}
[1] "terça"
\end{Soutput}
\begin{Sinput}
> strftime(dt, format = "%a") # Dia da semana abreviado.
\end{Sinput}
\begin{Soutput}
[1] "Ter"
\end{Soutput}
\begin{Sinput}
> strftime(dt, format = "%w") # Número do dia da semana (Seg = 1).
\end{Sinput}
\begin{Soutput}
[1] "2"
\end{Soutput}
\begin{Sinput}
> strftime(dt, format = "%B") # Nome do mês.
\end{Sinput}
\begin{Soutput}
[1] "março"
\end{Soutput}
\begin{Sinput}
> strftime(dt, format = "%b") # Nome do mês abreviado.
\end{Sinput}
\begin{Soutput}
[1] "Mar"
\end{Soutput}
\begin{Sinput}
> strftime(dt, format = "%m") # Número do mês.
\end{Sinput}
\begin{Soutput}
[1] "03"
\end{Soutput}
\begin{Sinput}
> strftime(dt, format = "%d") # Dia do mês.
\end{Sinput}
\begin{Soutput}
[1] "20"
\end{Soutput}
\begin{Sinput}
> strftime(dt, format = "%j") # Dia do ano, de 1 até 365/366.
\end{Sinput}
\begin{Soutput}
[1] "079"
\end{Soutput}
\begin{Sinput}
> strftime(dt, format = "%Y") # Ano.
\end{Sinput}
\begin{Soutput}
[1] "2018"
\end{Soutput}
\begin{Sinput}
> strftime(dt, format = "%y") # Ano só a dezena.
\end{Sinput}
\begin{Soutput}
[1] "18"
\end{Soutput}
\begin{Sinput}
> strftime(dt, format = "%H") # Hora.
\end{Sinput}
\begin{Soutput}
[1] "22"
\end{Soutput}
\begin{Sinput}
> strftime(dt, format = "%M") # Minuto.
\end{Sinput}
\begin{Soutput}
[1] "00"
\end{Soutput}
\begin{Sinput}
> strftime(dt, format = "%S") # Segundo.
\end{Sinput}
\begin{Soutput}
[1] "00"
\end{Soutput}
\begin{Sinput}
> agora <- as.POSIXct(Sys.time())
> falta <- dt - agora
> nd <- as.numeric(falta, units = "days")
> if (nd > 2) {
+     prazo <- sprintf(
+         "Faltam %d dias e %0.1f horas para entregar o trabalho",
+         nd %/% 1,
+         (nd %% 1) * 24)
+ } else if (nd > 1) {
+     prazo <- sprintf(
+         "Faltam 1 dia e %0.1f horas para entregar o trabalho",
+         (nd %% 1) * 24)
+ } else {
+     prazo <- sprintf(
+         "Falta %0.1f horas para entregar o trabalho",
+         nd * 24)
+ }
\end{Sinput}
\end{Schunk}

\tc{Faltam 6 dias e 5.3 horas para entregar o trabalho}.

\end{document}
