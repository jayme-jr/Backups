\documentclass[]{article}
\usepackage{lmodern}
\usepackage{amssymb,amsmath}
\usepackage{ifxetex,ifluatex}
\usepackage{fixltx2e} % provides \textsubscript
\ifnum 0\ifxetex 1\fi\ifluatex 1\fi=0 % if pdftex
  \usepackage[T1]{fontenc}
  \usepackage[utf8]{inputenc}
\else % if luatex or xelatex
  \ifxetex
    \usepackage{mathspec}
  \else
    \usepackage{fontspec}
  \fi
  \defaultfontfeatures{Ligatures=TeX,Scale=MatchLowercase}
\fi
% use upquote if available, for straight quotes in verbatim environments
\IfFileExists{upquote.sty}{\usepackage{upquote}}{}
% use microtype if available
\IfFileExists{microtype.sty}{%
\usepackage{microtype}
\UseMicrotypeSet[protrusion]{basicmath} % disable protrusion for tt fonts
}{}
\usepackage[margin=1in]{geometry}
\usepackage{hyperref}
\hypersetup{unicode=true,
            pdftitle={Trabalho 16/08/2018 Métodos de Amostragem},
            pdfauthor={Jayme Junior},
            pdfborder={0 0 0},
            breaklinks=true}
\urlstyle{same}  % don't use monospace font for urls
\usepackage{color}
\usepackage{fancyvrb}
\newcommand{\VerbBar}{|}
\newcommand{\VERB}{\Verb[commandchars=\\\{\}]}
\DefineVerbatimEnvironment{Highlighting}{Verbatim}{commandchars=\\\{\}}
% Add ',fontsize=\small' for more characters per line
\usepackage{framed}
\definecolor{shadecolor}{RGB}{248,248,248}
\newenvironment{Shaded}{\begin{snugshade}}{\end{snugshade}}
\newcommand{\KeywordTok}[1]{\textcolor[rgb]{0.13,0.29,0.53}{\textbf{#1}}}
\newcommand{\DataTypeTok}[1]{\textcolor[rgb]{0.13,0.29,0.53}{#1}}
\newcommand{\DecValTok}[1]{\textcolor[rgb]{0.00,0.00,0.81}{#1}}
\newcommand{\BaseNTok}[1]{\textcolor[rgb]{0.00,0.00,0.81}{#1}}
\newcommand{\FloatTok}[1]{\textcolor[rgb]{0.00,0.00,0.81}{#1}}
\newcommand{\ConstantTok}[1]{\textcolor[rgb]{0.00,0.00,0.00}{#1}}
\newcommand{\CharTok}[1]{\textcolor[rgb]{0.31,0.60,0.02}{#1}}
\newcommand{\SpecialCharTok}[1]{\textcolor[rgb]{0.00,0.00,0.00}{#1}}
\newcommand{\StringTok}[1]{\textcolor[rgb]{0.31,0.60,0.02}{#1}}
\newcommand{\VerbatimStringTok}[1]{\textcolor[rgb]{0.31,0.60,0.02}{#1}}
\newcommand{\SpecialStringTok}[1]{\textcolor[rgb]{0.31,0.60,0.02}{#1}}
\newcommand{\ImportTok}[1]{#1}
\newcommand{\CommentTok}[1]{\textcolor[rgb]{0.56,0.35,0.01}{\textit{#1}}}
\newcommand{\DocumentationTok}[1]{\textcolor[rgb]{0.56,0.35,0.01}{\textbf{\textit{#1}}}}
\newcommand{\AnnotationTok}[1]{\textcolor[rgb]{0.56,0.35,0.01}{\textbf{\textit{#1}}}}
\newcommand{\CommentVarTok}[1]{\textcolor[rgb]{0.56,0.35,0.01}{\textbf{\textit{#1}}}}
\newcommand{\OtherTok}[1]{\textcolor[rgb]{0.56,0.35,0.01}{#1}}
\newcommand{\FunctionTok}[1]{\textcolor[rgb]{0.00,0.00,0.00}{#1}}
\newcommand{\VariableTok}[1]{\textcolor[rgb]{0.00,0.00,0.00}{#1}}
\newcommand{\ControlFlowTok}[1]{\textcolor[rgb]{0.13,0.29,0.53}{\textbf{#1}}}
\newcommand{\OperatorTok}[1]{\textcolor[rgb]{0.81,0.36,0.00}{\textbf{#1}}}
\newcommand{\BuiltInTok}[1]{#1}
\newcommand{\ExtensionTok}[1]{#1}
\newcommand{\PreprocessorTok}[1]{\textcolor[rgb]{0.56,0.35,0.01}{\textit{#1}}}
\newcommand{\AttributeTok}[1]{\textcolor[rgb]{0.77,0.63,0.00}{#1}}
\newcommand{\RegionMarkerTok}[1]{#1}
\newcommand{\InformationTok}[1]{\textcolor[rgb]{0.56,0.35,0.01}{\textbf{\textit{#1}}}}
\newcommand{\WarningTok}[1]{\textcolor[rgb]{0.56,0.35,0.01}{\textbf{\textit{#1}}}}
\newcommand{\AlertTok}[1]{\textcolor[rgb]{0.94,0.16,0.16}{#1}}
\newcommand{\ErrorTok}[1]{\textcolor[rgb]{0.64,0.00,0.00}{\textbf{#1}}}
\newcommand{\NormalTok}[1]{#1}
\usepackage{longtable,booktabs}
\usepackage{graphicx,grffile}
\makeatletter
\def\maxwidth{\ifdim\Gin@nat@width>\linewidth\linewidth\else\Gin@nat@width\fi}
\def\maxheight{\ifdim\Gin@nat@height>\textheight\textheight\else\Gin@nat@height\fi}
\makeatother
% Scale images if necessary, so that they will not overflow the page
% margins by default, and it is still possible to overwrite the defaults
% using explicit options in \includegraphics[width, height, ...]{}
\setkeys{Gin}{width=\maxwidth,height=\maxheight,keepaspectratio}
\IfFileExists{parskip.sty}{%
\usepackage{parskip}
}{% else
\setlength{\parindent}{0pt}
\setlength{\parskip}{6pt plus 2pt minus 1pt}
}
\setlength{\emergencystretch}{3em}  % prevent overfull lines
\providecommand{\tightlist}{%
  \setlength{\itemsep}{0pt}\setlength{\parskip}{0pt}}
\setcounter{secnumdepth}{0}
% Redefines (sub)paragraphs to behave more like sections
\ifx\paragraph\undefined\else
\let\oldparagraph\paragraph
\renewcommand{\paragraph}[1]{\oldparagraph{#1}\mbox{}}
\fi
\ifx\subparagraph\undefined\else
\let\oldsubparagraph\subparagraph
\renewcommand{\subparagraph}[1]{\oldsubparagraph{#1}\mbox{}}
\fi

%%% Use protect on footnotes to avoid problems with footnotes in titles
\let\rmarkdownfootnote\footnote%
\def\footnote{\protect\rmarkdownfootnote}

%%% Change title format to be more compact
\usepackage{titling}

% Create subtitle command for use in maketitle
\newcommand{\subtitle}[1]{
  \posttitle{
    \begin{center}\large#1\end{center}
    }
}

\setlength{\droptitle}{-2em}

  \title{Trabalho 16/08/2018 Métodos de Amostragem}
    \pretitle{\vspace{\droptitle}\centering\huge}
  \posttitle{\par}
    \author{Jayme Junior}
    \preauthor{\centering\large\emph}
  \postauthor{\par}
    \date{}
    \predate{}\postdate{}
  

\begin{document}
\maketitle

\section{Para n = 2}\label{para-n-2}

\begin{longtable}[]{@{}llll@{}}
\toprule
Amostra & ys & P(ys) & ys.P(ys)\tabularnewline
\midrule
\endhead
2,4 & 3 & 1/15 & 0.2\tabularnewline
2,5 & 3.5 & 1/15 & 0.233333333333333\tabularnewline
2,7 & 4.5 & 1/15 & 0.3\tabularnewline
2,8 & 5 & 1/15 & 0.333333333333333\tabularnewline
2,9 & 5.5 & 1/15 & 0.366666666666667\tabularnewline
4,5 & 4.5 & 1/15 & 0.3\tabularnewline
4,7 & 5.5 & 1/15 & 0.366666666666667\tabularnewline
4,8 & 6 & 1/15 & 0.4\tabularnewline
4,9 & 6.5 & 1/15 & 0.433333333333333\tabularnewline
5,7 & 6 & 1/15 & 0.4\tabularnewline
5,8 & 6.5 & 1/15 & 0.433333333333333\tabularnewline
5,9 & 7 & 1/15 & 0.466666666666667\tabularnewline
7,8 & 7.5 & 1/15 & 0.5\tabularnewline
7,9 & 8 & 1/15 & 0.533333333333333\tabularnewline
8,9 & 8.5 & 1/15 & 0.566666666666667\tabularnewline
\bottomrule
\end{longtable}

{[}1{]} ``Comparando média amostral n=2: 5.83 e populacional: 5.83 temos
que são iguais.''

\section{Para n = 3}\label{para-n-3}

\begin{longtable}[]{@{}llll@{}}
\toprule
Amostra & ys & P(ys) & ys.P(ys)\tabularnewline
\midrule
\endhead
2,4,5 & 3.66666666666667 & 1/20 & 0.183333333333333\tabularnewline
2,4,7 & 4.33333333333333 & 1/20 & 0.216666666666667\tabularnewline
2,4,8 & 4.66666666666667 & 1/20 & 0.233333333333333\tabularnewline
2,4,9 & 5 & 1/20 & 0.25\tabularnewline
2,5,7 & 4.66666666666667 & 1/20 & 0.233333333333333\tabularnewline
2,5,8 & 5 & 1/20 & 0.25\tabularnewline
2,5,9 & 5.33333333333333 & 1/20 & 0.266666666666667\tabularnewline
2,7,8 & 5.66666666666667 & 1/20 & 0.283333333333333\tabularnewline
2,7,9 & 6 & 1/20 & 0.3\tabularnewline
2,8,9 & 6.33333333333333 & 1/20 & 0.316666666666667\tabularnewline
4,5,7 & 5.33333333333333 & 1/20 & 0.266666666666667\tabularnewline
4,5,8 & 5.66666666666667 & 1/20 & 0.283333333333333\tabularnewline
4,5,9 & 6 & 1/20 & 0.3\tabularnewline
4,7,8 & 6.33333333333333 & 1/20 & 0.316666666666667\tabularnewline
4,7,9 & 6.66666666666667 & 1/20 & 0.333333333333333\tabularnewline
4,8,9 & 7 & 1/20 & 0.35\tabularnewline
5,7,8 & 6.66666666666667 & 1/20 & 0.333333333333333\tabularnewline
5,7,9 & 7 & 1/20 & 0.35\tabularnewline
5,8,9 & 7.33333333333333 & 1/20 & 0.366666666666667\tabularnewline
7,8,9 & 8 & 1/20 & 0.4\tabularnewline
\bottomrule
\end{longtable}

{[}1{]} ``Comparando média amostral n=3: 5.83 e populacional: 5.83 temos
que são iguais.''

\section{Para n = 4}\label{para-n-4}

\begin{longtable}[]{@{}llll@{}}
\toprule
Amostra & ys & P(ys) & ys.P(ys)\tabularnewline
\midrule
\endhead
2,4,5,7 & 4.5 & 1/15 & 0.3\tabularnewline
2,4,5,8 & 4.75 & 1/15 & 0.316666666666667\tabularnewline
2,4,5,9 & 5 & 1/15 & 0.333333333333333\tabularnewline
2,4,7,8 & 5.25 & 1/15 & 0.35\tabularnewline
2,4,7,9 & 5.5 & 1/15 & 0.366666666666667\tabularnewline
2,4,8,9 & 5.75 & 1/15 & 0.383333333333333\tabularnewline
2,5,7,8 & 5.5 & 1/15 & 0.366666666666667\tabularnewline
2,5,7,9 & 5.75 & 1/15 & 0.383333333333333\tabularnewline
2,5,8,9 & 6 & 1/15 & 0.4\tabularnewline
2,7,8,9 & 6.5 & 1/15 & 0.433333333333333\tabularnewline
4,5,7,8 & 6 & 1/15 & 0.4\tabularnewline
4,5,7,9 & 6.25 & 1/15 & 0.416666666666667\tabularnewline
4,5,8,9 & 6.5 & 1/15 & 0.433333333333333\tabularnewline
4,7,8,9 & 7 & 1/15 & 0.466666666666667\tabularnewline
5,7,8,9 & 7.25 & 1/15 & 0.483333333333333\tabularnewline
\bottomrule
\end{longtable}

{[}1{]} ``Comparando média amostral n=4: 5.83 e populacional: 5.83 temos
que são iguais.''

\section{Para todas as amostras n=2, n=3 e
n=4}\label{para-todas-as-amostras-n2-n3-e-n4}

\begin{longtable}[]{@{}llll@{}}
\toprule
Amostra & ys & P(ys) & ys.P(ys)\tabularnewline
\midrule
\endhead
2,4 & 3 & 1/15 & 0.2\tabularnewline
2,5 & 3.5 & 1/15 & 0.233333333333333\tabularnewline
2,7 & 4.5 & 1/15 & 0.3\tabularnewline
2,8 & 5 & 1/15 & 0.333333333333333\tabularnewline
2,9 & 5.5 & 1/15 & 0.366666666666667\tabularnewline
4,5 & 4.5 & 1/15 & 0.3\tabularnewline
4,7 & 5.5 & 1/15 & 0.366666666666667\tabularnewline
4,8 & 6 & 1/15 & 0.4\tabularnewline
4,9 & 6.5 & 1/15 & 0.433333333333333\tabularnewline
5,7 & 6 & 1/15 & 0.4\tabularnewline
5,8 & 6.5 & 1/15 & 0.433333333333333\tabularnewline
5,9 & 7 & 1/15 & 0.466666666666667\tabularnewline
7,8 & 7.5 & 1/15 & 0.5\tabularnewline
7,9 & 8 & 1/15 & 0.533333333333333\tabularnewline
8,9 & 8.5 & 1/15 & 0.566666666666667\tabularnewline
2,4,5 & 3.66666666666667 & 1/20 & 0.183333333333333\tabularnewline
2,4,7 & 4.33333333333333 & 1/20 & 0.216666666666667\tabularnewline
2,4,8 & 4.66666666666667 & 1/20 & 0.233333333333333\tabularnewline
2,4,9 & 5 & 1/20 & 0.25\tabularnewline
2,5,7 & 4.66666666666667 & 1/20 & 0.233333333333333\tabularnewline
2,5,8 & 5 & 1/20 & 0.25\tabularnewline
2,5,9 & 5.33333333333333 & 1/20 & 0.266666666666667\tabularnewline
2,7,8 & 5.66666666666667 & 1/20 & 0.283333333333333\tabularnewline
2,7,9 & 6 & 1/20 & 0.3\tabularnewline
2,8,9 & 6.33333333333333 & 1/20 & 0.316666666666667\tabularnewline
4,5,7 & 5.33333333333333 & 1/20 & 0.266666666666667\tabularnewline
4,5,8 & 5.66666666666667 & 1/20 & 0.283333333333333\tabularnewline
4,5,9 & 6 & 1/20 & 0.3\tabularnewline
4,7,8 & 6.33333333333333 & 1/20 & 0.316666666666667\tabularnewline
4,7,9 & 6.66666666666667 & 1/20 & 0.333333333333333\tabularnewline
4,8,9 & 7 & 1/20 & 0.35\tabularnewline
5,7,8 & 6.66666666666667 & 1/20 & 0.333333333333333\tabularnewline
5,7,9 & 7 & 1/20 & 0.35\tabularnewline
5,8,9 & 7.33333333333333 & 1/20 & 0.366666666666667\tabularnewline
7,8,9 & 8 & 1/20 & 0.4\tabularnewline
2,4,5,7 & 4.5 & 1/15 & 0.3\tabularnewline
2,4,5,8 & 4.75 & 1/15 & 0.316666666666667\tabularnewline
2,4,5,9 & 5 & 1/15 & 0.333333333333333\tabularnewline
2,4,7,8 & 5.25 & 1/15 & 0.35\tabularnewline
2,4,7,9 & 5.5 & 1/15 & 0.366666666666667\tabularnewline
2,4,8,9 & 5.75 & 1/15 & 0.383333333333333\tabularnewline
2,5,7,8 & 5.5 & 1/15 & 0.366666666666667\tabularnewline
2,5,7,9 & 5.75 & 1/15 & 0.383333333333333\tabularnewline
2,5,8,9 & 6 & 1/15 & 0.4\tabularnewline
2,7,8,9 & 6.5 & 1/15 & 0.433333333333333\tabularnewline
4,5,7,8 & 6 & 1/15 & 0.4\tabularnewline
4,5,7,9 & 6.25 & 1/15 & 0.416666666666667\tabularnewline
4,5,8,9 & 6.5 & 1/15 & 0.433333333333333\tabularnewline
4,7,8,9 & 7 & 1/15 & 0.466666666666667\tabularnewline
5,7,8,9 & 7.25 & 1/15 & 0.483333333333333\tabularnewline
\bottomrule
\end{longtable}

\section{Trabalho 2 - 23/08}\label{trabalho-2---2308}

\subsection{n=2}\label{n2}

\begin{Shaded}
\begin{Highlighting}[]
\NormalTok{Y2 <-}\StringTok{ }\KeywordTok{sum}\NormalTok{(yspys2)}
\NormalTok{Y <-}\StringTok{ }\KeywordTok{mean}\NormalTok{(x)}
\KeywordTok{sum}\NormalTok{((ys2 }\OperatorTok{-}\StringTok{ }\NormalTok{Y)}\OperatorTok{^}\DecValTok{2}\NormalTok{) }\OperatorTok{*}\StringTok{ }\NormalTok{(}\DecValTok{1}\OperatorTok{/}\DecValTok{15}\NormalTok{)}
\end{Highlighting}
\end{Shaded}

\begin{verbatim}
## [1] 2.322222
\end{verbatim}

\begin{Shaded}
\begin{Highlighting}[]
\NormalTok{((}\DecValTok{6}\OperatorTok{-}\DecValTok{2}\NormalTok{)}\OperatorTok{/}\DecValTok{6}\NormalTok{) }\OperatorTok{*}\StringTok{ }\NormalTok{(}\KeywordTok{sum}\NormalTok{(((ys2 }\OperatorTok{-}\StringTok{ }\NormalTok{Y)}\OperatorTok{^}\DecValTok{2}\NormalTok{)}\OperatorTok{/}\DecValTok{5}\NormalTok{)}\OperatorTok{/}\DecValTok{2}\NormalTok{)}
\end{Highlighting}
\end{Shaded}

\begin{verbatim}
## [1] 2.322222
\end{verbatim}

\subsection{n=3}\label{n3}

\begin{Shaded}
\begin{Highlighting}[]
\NormalTok{Y <-}\StringTok{ }\KeywordTok{mean}\NormalTok{(x)}
\KeywordTok{sum}\NormalTok{((ys3 }\OperatorTok{-}\StringTok{ }\NormalTok{Y)}\OperatorTok{^}\DecValTok{2}\NormalTok{) }\OperatorTok{*}\StringTok{ }\NormalTok{(}\DecValTok{1}\OperatorTok{/}\DecValTok{20}\NormalTok{)}
\end{Highlighting}
\end{Shaded}

\begin{verbatim}
## [1] 1.161111
\end{verbatim}

\begin{Shaded}
\begin{Highlighting}[]
\NormalTok{((}\DecValTok{6}\OperatorTok{-}\DecValTok{3}\NormalTok{)}\OperatorTok{/}\DecValTok{6}\NormalTok{) }\OperatorTok{*}\StringTok{ }\NormalTok{(}\KeywordTok{sum}\NormalTok{(((ys3 }\OperatorTok{-}\StringTok{ }\NormalTok{Y)}\OperatorTok{^}\DecValTok{2}\NormalTok{)}\OperatorTok{/}\DecValTok{5}\NormalTok{) }\OperatorTok{*}\StringTok{ }\NormalTok{(}\DecValTok{1}\OperatorTok{/}\DecValTok{3}\NormalTok{))}
\end{Highlighting}
\end{Shaded}

\begin{verbatim}
## [1] 0.7740741
\end{verbatim}

\subsection{n=4}\label{n4}

\begin{Shaded}
\begin{Highlighting}[]
\NormalTok{Y <-}\StringTok{ }\KeywordTok{mean}\NormalTok{(x)}
\KeywordTok{sum}\NormalTok{((ys4 }\OperatorTok{-}\StringTok{ }\NormalTok{Y)}\OperatorTok{^}\DecValTok{2}\NormalTok{) }\OperatorTok{*}\StringTok{ }\NormalTok{(}\DecValTok{1}\OperatorTok{/}\DecValTok{15}\NormalTok{)}
\end{Highlighting}
\end{Shaded}

\begin{verbatim}
## [1] 0.5805556
\end{verbatim}

\begin{Shaded}
\begin{Highlighting}[]
\NormalTok{((}\DecValTok{6}\OperatorTok{-}\DecValTok{4}\NormalTok{)}\OperatorTok{/}\DecValTok{6}\NormalTok{) }\OperatorTok{*}\StringTok{ }\NormalTok{(}\KeywordTok{sum}\NormalTok{(((ys4 }\OperatorTok{-}\StringTok{ }\NormalTok{Y)}\OperatorTok{^}\DecValTok{2}\NormalTok{)}\OperatorTok{/}\DecValTok{5}\NormalTok{) }\OperatorTok{*}\StringTok{ }\NormalTok{(}\DecValTok{1}\OperatorTok{/}\DecValTok{4}\NormalTok{))}
\end{Highlighting}
\end{Shaded}

\begin{verbatim}
## [1] 0.1451389
\end{verbatim}


\end{document}
