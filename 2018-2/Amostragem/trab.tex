\documentclass[]{article}
\usepackage{lmodern}
\usepackage{amssymb,amsmath}
\usepackage{ifxetex,ifluatex}
\usepackage{fixltx2e} % provides \textsubscript
\ifnum 0\ifxetex 1\fi\ifluatex 1\fi=0 % if pdftex
  \usepackage[T1]{fontenc}
  \usepackage[utf8]{inputenc}
\else % if luatex or xelatex
  \ifxetex
    \usepackage{mathspec}
  \else
    \usepackage{fontspec}
  \fi
  \defaultfontfeatures{Ligatures=TeX,Scale=MatchLowercase}
\fi
% use upquote if available, for straight quotes in verbatim environments
\IfFileExists{upquote.sty}{\usepackage{upquote}}{}
% use microtype if available
\IfFileExists{microtype.sty}{%
\usepackage{microtype}
\UseMicrotypeSet[protrusion]{basicmath} % disable protrusion for tt fonts
}{}
\usepackage[margin=1in]{geometry}
\usepackage{hyperref}
\hypersetup{unicode=true,
            pdftitle={Trabalho de Métodos de Amostragem},
            pdfauthor={Jayme Gomes dos Santos Junior},
            pdfborder={0 0 0},
            breaklinks=true}
\urlstyle{same}  % don't use monospace font for urls
\usepackage{longtable,booktabs}
\usepackage{graphicx,grffile}
\makeatletter
\def\maxwidth{\ifdim\Gin@nat@width>\linewidth\linewidth\else\Gin@nat@width\fi}
\def\maxheight{\ifdim\Gin@nat@height>\textheight\textheight\else\Gin@nat@height\fi}
\makeatother
% Scale images if necessary, so that they will not overflow the page
% margins by default, and it is still possible to overwrite the defaults
% using explicit options in \includegraphics[width, height, ...]{}
\setkeys{Gin}{width=\maxwidth,height=\maxheight,keepaspectratio}
\IfFileExists{parskip.sty}{%
\usepackage{parskip}
}{% else
\setlength{\parindent}{0pt}
\setlength{\parskip}{6pt plus 2pt minus 1pt}
}
\setlength{\emergencystretch}{3em}  % prevent overfull lines
\providecommand{\tightlist}{%
  \setlength{\itemsep}{0pt}\setlength{\parskip}{0pt}}
\setcounter{secnumdepth}{5}
% Redefines (sub)paragraphs to behave more like sections
\ifx\paragraph\undefined\else
\let\oldparagraph\paragraph
\renewcommand{\paragraph}[1]{\oldparagraph{#1}\mbox{}}
\fi
\ifx\subparagraph\undefined\else
\let\oldsubparagraph\subparagraph
\renewcommand{\subparagraph}[1]{\oldsubparagraph{#1}\mbox{}}
\fi

%%% Use protect on footnotes to avoid problems with footnotes in titles
\let\rmarkdownfootnote\footnote%
\def\footnote{\protect\rmarkdownfootnote}

%%% Change title format to be more compact
\usepackage{titling}

% Create subtitle command for use in maketitle
\newcommand{\subtitle}[1]{
  \posttitle{
    \begin{center}\large#1\end{center}
    }
}

\setlength{\droptitle}{-2em}

  \title{Trabalho de Métodos de Amostragem}
    \pretitle{\vspace{\droptitle}\centering\huge}
  \posttitle{\par}
    \author{Jayme Gomes dos Santos Junior}
    \preauthor{\centering\large\emph}
  \postauthor{\par}
      \predate{\centering\large\emph}
  \postdate{\par}
    \date{Curitiba 2018}

\usepackage{booktabs}
\usepackage{longtable}
\usepackage{array}
\usepackage{multirow}
\usepackage[table]{xcolor}
\usepackage{wrapfig}
\usepackage{float}
\usepackage{colortbl}
\usepackage{pdflscape}
\usepackage{tabu}
\usepackage{threeparttable}
\usepackage{threeparttablex}
\usepackage[normalem]{ulem}
\usepackage{makecell}

\usepackage[brazil]{babel}

\begin{document}
\maketitle

{
\setcounter{tocdepth}{3}
\tableofcontents
}
\newpage

\section{Trabalho 1}\label{trabalho-1}

\subsection{Teorema 1}\label{teorema-1}

Tendo uma população de tamanho \textbf{N}, ao extrair uma amostra de
tamanho \textbf{n}, temos que o estimador não viezado para
\(\overline{Y}\) é: \[
\overline{y} = \sum_{i=1}^n\dfrac{y_i}{n}
\]

\subsection{Exercício 1}\label{exercicio-1}

\begin{longtable}[]{@{}cl@{}}
\caption{População de Tamanho N = 6}\tabularnewline
\toprule
População & Valores\tabularnewline
\midrule
\endfirsthead
\toprule
População & Valores\tabularnewline
\midrule
\endhead
y1 & 2\tabularnewline
y2 & 4\tabularnewline
y3 & 5\tabularnewline
y4 & 7\tabularnewline
y5 & 8\tabularnewline
y6 & 9\tabularnewline
\bottomrule
\end{longtable}

Com base na população dada na tabela 1, encontrar a média de todas as
combinações possíveis de amostras de tamanhos: n = 2, n = 3 e n = 4.

\subsubsection{Resolução}\label{resolucao}

Sabendo que a média populacional é dada por:

\[
\overline{Y} = \sum_{i=1}^N\dfrac{y_i}{N} = \dfrac{2+4+5+7+8+9}{6} = 5.83
\] O objetivo é mostrar que a média de todas as combinações de amostras
possíveis terão o mesmo valor da média populacional.

\emph{OBS:} ys = médias amostrais, P(ys) = probabilidade de cada amostra
acontecer e ys.P(ys) = média amostral multiplicada pela sua
probabilidade e a média amostral é calculada através do cálculo

\[
\sum_{i=1}^nys_i.P(ys_i)
\]

\paragraph{Para n = 2, temos:}\label{para-n-2-temos}

A quantidade de amostras possíveis de tamanho 2 = \({6\choose 2}\) = 15.

\newpage

\begin{longtable}[]{@{}clcc@{}}
\caption{Amostras Possíveis com n = 2}\tabularnewline
\toprule
Amostra & ys & P(ys) & ys.P(ys)\tabularnewline
\midrule
\endfirsthead
\toprule
Amostra & ys & P(ys) & ys.P(ys)\tabularnewline
\midrule
\endhead
2,4 & 3 & 1/15 & 0.2\tabularnewline
2,5 & 3.5 & 1/15 & 0.23\tabularnewline
2,7 & 4.5 & 1/15 & 0.3\tabularnewline
2,8 & 5 & 1/15 & 0.33\tabularnewline
2,9 & 5.5 & 1/15 & 0.37\tabularnewline
4,5 & 4.5 & 1/15 & 0.3\tabularnewline
4,7 & 5.5 & 1/15 & 0.37\tabularnewline
4,8 & 6 & 1/15 & 0.4\tabularnewline
4,9 & 6.5 & 1/15 & 0.43\tabularnewline
5,7 & 6 & 1/15 & 0.4\tabularnewline
5,8 & 6.5 & 1/15 & 0.43\tabularnewline
5,9 & 7 & 1/15 & 0.47\tabularnewline
7,8 & 7.5 & 1/15 & 0.5\tabularnewline
7,9 & 8 & 1/15 & 0.53\tabularnewline
8,9 & 8.5 & 1/15 & 0.57\tabularnewline
\bottomrule
\end{longtable}

Comparando a média amostral com n = 2: 5.83 e populacional: 5.83, temos
que são iguai.

\paragraph{Para n = 3, temos:}\label{para-n-3-temos}

A quantidade de amostras possíveis de tamanho 2 = \({6\choose 3}\) = 20.

\begin{longtable}[]{@{}clcc@{}}
\caption{Amostras Possíveis com n = 3}\tabularnewline
\toprule
Amostra & ys & P(ys) & ys.P(ys)\tabularnewline
\midrule
\endfirsthead
\toprule
Amostra & ys & P(ys) & ys.P(ys)\tabularnewline
\midrule
\endhead
2,4,5 & 3.67 & 1/20 & 0.18\tabularnewline
2,4,7 & 4.33 & 1/20 & 0.22\tabularnewline
2,4,8 & 4.67 & 1/20 & 0.23\tabularnewline
2,4,9 & 5 & 1/20 & 0.25\tabularnewline
2,5,7 & 4.67 & 1/20 & 0.23\tabularnewline
2,5,8 & 5 & 1/20 & 0.25\tabularnewline
2,5,9 & 5.33 & 1/20 & 0.27\tabularnewline
2,7,8 & 5.67 & 1/20 & 0.28\tabularnewline
2,7,9 & 6 & 1/20 & 0.3\tabularnewline
2,8,9 & 6.33 & 1/20 & 0.32\tabularnewline
4,5,7 & 5.33 & 1/20 & 0.27\tabularnewline
4,5,8 & 5.67 & 1/20 & 0.28\tabularnewline
4,5,9 & 6 & 1/20 & 0.3\tabularnewline
4,7,8 & 6.33 & 1/20 & 0.32\tabularnewline
4,7,9 & 6.67 & 1/20 & 0.33\tabularnewline
4,8,9 & 7 & 1/20 & 0.35\tabularnewline
5,7,8 & 6.67 & 1/20 & 0.33\tabularnewline
5,7,9 & 7 & 1/20 & 0.35\tabularnewline
5,8,9 & 7.33 & 1/20 & 0.37\tabularnewline
7,8,9 & 8 & 1/20 & 0.4\tabularnewline
\bottomrule
\end{longtable}

Comparando a média amostral com n = 3: 5.83 e populacional: 5.83, temos
que são iguai.

\paragraph{Para n = 4, temos:}\label{para-n-4-temos}

A quantidade de amostras possíveis de tamanho 4 = \({6\choose 4}\) = 15.

\begin{longtable}[]{@{}clcc@{}}
\caption{Amostras Possíveis com n = 4}\tabularnewline
\toprule
Amostra & ys & P(ys) & ys.P(ys)\tabularnewline
\midrule
\endfirsthead
\toprule
Amostra & ys & P(ys) & ys.P(ys)\tabularnewline
\midrule
\endhead
2,4,5,7 & 4.5 & 1/15 & 0.3\tabularnewline
2,4,5,8 & 4.75 & 1/15 & 0.32\tabularnewline
2,4,5,9 & 5 & 1/15 & 0.33\tabularnewline
2,4,7,8 & 5.25 & 1/15 & 0.35\tabularnewline
2,4,7,9 & 5.5 & 1/15 & 0.37\tabularnewline
2,4,8,9 & 5.75 & 1/15 & 0.38\tabularnewline
2,5,7,8 & 5.5 & 1/15 & 0.37\tabularnewline
2,5,7,9 & 5.75 & 1/15 & 0.38\tabularnewline
2,5,8,9 & 6 & 1/15 & 0.4\tabularnewline
2,7,8,9 & 6.5 & 1/15 & 0.43\tabularnewline
4,5,7,8 & 6 & 1/15 & 0.4\tabularnewline
4,5,7,9 & 6.25 & 1/15 & 0.42\tabularnewline
4,5,8,9 & 6.5 & 1/15 & 0.43\tabularnewline
4,7,8,9 & 7 & 1/15 & 0.47\tabularnewline
5,7,8,9 & 7.25 & 1/15 & 0.48\tabularnewline
\bottomrule
\end{longtable}

Comparando a média amostral com n = 4: 5.83 e populacional: 5.83, temos
que são iguai.

\section{Trabalho 2}\label{trabalho-2}

\subsection{Teorema 2}\label{teorema-2}

Seja \textbf{Y} a variável de interesse e \textbf{N} o tamanho da
população, considere uma amostra de tamanho \textbf{n} extraída pelo
processo de amostragem aleatória. A variãncia da média amostral será
dada por:

\[
V[\overline{y}]=E[(\overline{y}-\overline{Y})^2]=\bigg[ \dfrac{N-n}{N}\bigg] \dfrac{S^2}{n}
\] ou

\[
V[\overline{y}]=\sum_{i=1}^n(\overline{y_i}-\overline{Y})^2p(ys)_i
\]

\subsection{Exercício 2}\label{exercicio-2}

Usando a população do Exerecício 1, calcular a variância da média
amostral pelas duas formas para amostras n = 2, n = 3 e n = 4 com o
objetivo de mostrar que os resultados são os mesmos usando as duas
maneiras para cada tamanho de amostra.

Considerando a tabela 1, temos que:

\[
S^2=\sum_{i=1}^N\dfrac{(y_i-\overline{Y})^2}{N-1}=\dfrac{(2-5.83)^2}{6-1}+...+\dfrac{(9-5.83)^2}{6-1}\cong6.967
\]

\subsubsection{Resolução}\label{resolucao-1}

\paragraph{Para n = 2:}\label{para-n-2}

\textbf{1ª Forma:} \[
V[\overline{y}]=\bigg[\dfrac{N-n}{N}\bigg] \dfrac{S^2}{n}=\bigg[\dfrac{6-2}{6}\bigg]\dfrac{6.967}{2}\cong2.32
\]

\textbf{2ª Forma:} \[
V[\overline{y}]=\sum_{i=1}^n(\overline{y_i}-\overline{Y})^2p(ys)_i=\dfrac{(3-5.83)^2}{6-1}\dfrac{1}{15}+\dfrac{(3.5-5.83)^2}{6-1}\dfrac{1}{15}+...+\dfrac{(8.5-5,83)^2}{6-1}\dfrac{1}{15}\cong2.32
\]

\paragraph{Para n = 3:}\label{para-n-3}

\textbf{1ª Forma:} \[
V[\overline{y}]=\bigg[\dfrac{N-n}{N}\bigg] \dfrac{S^2}{n}=\bigg[\dfrac{6-3}{6}\bigg]\dfrac{6.967}{3}\cong1.16
\]

\textbf{2ª Forma:} \[
V[\overline{y}]=\sum_{i=1}^n(\overline{y_i}-\overline{Y})^2p(ys)_i=\dfrac{(3.67-5.83)^2}{6-1}\dfrac{1}{20}+\dfrac{(4.33-5.83)^2}{6-1}\dfrac{1}{20}+...+\dfrac{(8-5,83)^2}{6-1}\dfrac{1}{10}\cong1.16
\]

\paragraph{Para n = 4:}\label{para-n-4}

\textbf{1ª Forma:} \[
V[\overline{y}]=\bigg[\dfrac{N-n}{N}\bigg] \dfrac{S^2}{n}=\bigg[\dfrac{6-4}{6}\bigg]\dfrac{6.967}{4}\cong0.58
\]

\textbf{2ª Forma:} \[
V[\overline{y}]=\sum_{i=1}^n(\overline{y_i}-\overline{Y})^2p(ys)_i=\dfrac{(4.5-5.83)^2}{6-1}\dfrac{1}{15}+\dfrac{(4.75-5.83)^2}{6-1}\dfrac{1}{15}+...+\dfrac{(7.25-5,83)^2}{6-1}\dfrac{1}{15}\cong0.58
\]

Logo, está demonstrado que as duas formas de cálculo de variância geram
o mesmo resultado e que quanto mais aumenta o tamanho da amostra, mais
diminui a variãncia, como esperado.

\section{Trabalho 3}\label{trabalho-3}

\subsection{Teorema 3}\label{teorema-3}

Variânriãncia amostral é um estimador não viezado para a variância
populacional.

\[
E[s^2]=\sum_{i=1}^ns^2P(ys_i)=S^2
\]

Onde:

\[
s^2=\dfrac{(a_1-ys)^2+...+(a_n-ys)^2}{n-1}
\]

Sendo \(a_1, ..., a_n\) cada observação da amostra.

\subsection{Exercício 3}\label{exercicio-3}

Provar o teorema 3 utilizando as informações dos exercícios anteriores.
Sabemos pelo exercício 2 que: \(S^2=6,967\).

\subsubsection{Resolução}\label{resolucao-2}

\textbf{n = 2:}

\begin{longtable}[]{@{}cll@{}}
\caption{Variãncia amostral para n = 2}\tabularnewline
\toprule
Amostra & s² & s²P(ys)\tabularnewline
\midrule
\endfirsthead
\toprule
Amostra & s² & s²P(ys)\tabularnewline
\midrule
\endhead
2,4 & 2 & 0.13\tabularnewline
2,5 & 4.5 & 0.3\tabularnewline
2,7 & 12.5 & 0.83\tabularnewline
2,8 & 18 & 1.2\tabularnewline
2,9 & 24.5 & 1.63\tabularnewline
4,5 & 0.5 & 0.03\tabularnewline
4,7 & 4.5 & 0.3\tabularnewline
4,8 & 8 & 0.53\tabularnewline
4,9 & 12.5 & 0.83\tabularnewline
5,7 & 2 & 0.13\tabularnewline
5,8 & 4.5 & 0.3\tabularnewline
5,9 & 8 & 0.53\tabularnewline
7,8 & 0.5 & 0.03\tabularnewline
7,9 & 2 & 0.13\tabularnewline
8,9 & 0.5 & 0.03\tabularnewline
\bottomrule
\end{longtable}

Calculando \(E[s^2]\) = 6.967 = \(S^2\)

\textbf{n = 3:}

\begin{longtable}[]{@{}cll@{}}
\caption{Variãncia amostral para n = 3}\tabularnewline
\toprule
Amostra & s² & s²P(ys)\tabularnewline
\midrule
\endfirsthead
\toprule
Amostra & s² & s²P(ys)\tabularnewline
\midrule
\endhead
2,4,5 & 2.33 & 0.12\tabularnewline
2,4,7 & 6.33 & 0.32\tabularnewline
2,4,8 & 9.33 & 0.47\tabularnewline
2,4,9 & 13 & 0.65\tabularnewline
2,5,7 & 6.33 & 0.32\tabularnewline
2,5,8 & 9 & 0.45\tabularnewline
2,5,9 & 12.33 & 0.62\tabularnewline
2,7,8 & 10.33 & 0.52\tabularnewline
2,7,9 & 13 & 0.65\tabularnewline
2,8,9 & 14.33 & 0.72\tabularnewline
4,5,7 & 2.33 & 0.12\tabularnewline
4,5,8 & 4.33 & 0.22\tabularnewline
4,5,9 & 7 & 0.35\tabularnewline
4,7,8 & 4.33 & 0.22\tabularnewline
4,7,9 & 6.33 & 0.32\tabularnewline
4,8,9 & 7 & 0.35\tabularnewline
5,7,8 & 2.33 & 0.12\tabularnewline
5,7,9 & 4 & 0.2\tabularnewline
5,8,9 & 4.33 & 0.22\tabularnewline
7,8,9 & 1 & 0.05\tabularnewline
\bottomrule
\end{longtable}

Calculando \(E[s^2]\) = 6.967 = \(S^2\)

\textbf{n = 4:}

\begin{longtable}[]{@{}cll@{}}
\caption{Variãncia amostral para n = 4}\tabularnewline
\toprule
Amostra & s² & s²P(ys)\tabularnewline
\midrule
\endfirsthead
\toprule
Amostra & s² & s²P(ys)\tabularnewline
\midrule
\endhead
2,4,5,7 & 4.33 & 0.29\tabularnewline
2,4,5,8 & 6.25 & 0.42\tabularnewline
2,4,5,9 & 8.67 & 0.58\tabularnewline
2,4,7,8 & 7.58 & 0.51\tabularnewline
2,4,7,9 & 9.67 & 0.64\tabularnewline
2,4,8,9 & 10.92 & 0.73\tabularnewline
2,5,7,8 & 7 & 0.47\tabularnewline
2,5,7,9 & 8.92 & 0.59\tabularnewline
2,5,8,9 & 10 & 0.67\tabularnewline
2,7,8,9 & 9.67 & 0.64\tabularnewline
4,5,7,8 & 3.33 & 0.22\tabularnewline
4,5,7,9 & 4.92 & 0.33\tabularnewline
4,5,8,9 & 5.67 & 0.38\tabularnewline
4,7,8,9 & 4.67 & 0.31\tabularnewline
5,7,8,9 & 2.92 & 0.19\tabularnewline
\bottomrule
\end{longtable}

Calculando \(E[s^2]\) = 6.967 = \(S^2\)

\section{Trabalho 4}\label{trabalho-4}

\subsection{Intervalo de Confiança(IC)}\label{intervalo-de-confiancaic}

O cálculo do IC é feito atravéz da fórmula

\[
IC = \overline{Y}\pm Z_{1-\frac{\alpha}{2}}\sqrt{\dfrac{N-n}{N}}\dfrac{S}{\sqrt{n}}
\]

\subsection{Exercício 4}\label{exercicio-4}

Usando os resultados do Exercício 1, construir o intervalo de confiança
considerando n = 2, n = 3 e n = 4 e \(\alpha\) - 0.2 e verificar quantas
das médias amostrais se encontram dentro do IC.

\subsubsection{Resolução}\label{resolucao-3}

\textbf{n = 2}

\[
IC = 5.83\pm 2.33\sqrt{\dfrac{6-2}{6}}\dfrac{2.63}{\sqrt{2}}=5.83\pm 3.55 \rightarrow IC = (2.28;9.38)
\]

Com basae nos resultados da Tabela 2 é possível verificar que todas as
médias amostrais estão dentro do IC.

\textbf{n = 3}

\[
IC = 5.83\pm 2.33\sqrt{\dfrac{6-3}{6}}\dfrac{2.63}{\sqrt{3}}=5.83\pm 2.51 \rightarrow IC = (3.32;8.38)
\]

Com basae nos resultados da Tabela 3 é possível verificar que todas as
médias amostrais estão dentro do IC.

\textbf{n = 4}

\[
IC = 5.83\pm 2.33\sqrt{\dfrac{6-4}{6}}\dfrac{2.63}{\sqrt{4}}=5.83\pm 1.78 \rightarrow IC = (4.05;7.61)
\]

Com basae nos resultados da Tabela 4 é possível verificar que todas as
médias amostrais estão dentro do IC.

\section{Trabalho 5}\label{trabalho-5}

\subsection{Teorema 4}\label{teorema-4}

Para calcular o tamanho amostral de uma pesquisa é necessário usar uma
amostra piloto, com as informações desta amostra é possível calcular:

\[
n=\dfrac{n_0}{1+\frac{n_0}{N}} \rightarrow n_0=\bigg(\dfrac{Z_{1-\frac{\alpha}{2}}s_p}{r\overline{y}_p}\bigg)^2
\]

\emph{OBS:} \(s_p\) e \(\overline{y}_p\) são o desvio padrão e média da
amostra piloto

\subsection{Exercício}\label{exercicio}

Com base nos dados referentes a QI entregues em sala (Anexo 1),
verificar se o tamanho da amostra piloto na folha está adequado e
recalcular os parâmetros com o novo tamanho de amostra. Considerar r =
0.03

\subsubsection{Resolução}\label{resolucao-4}

\paragraph{Amostra Piloto}\label{amostra-piloto}

As 10 observações aleatórias para a amostra piloto foram colhidas usando
o número primo 11 e somando mais 11 para selecionar o pŕoximo, sempre
nas linhas da esquerda para a direita, gerando:

\begin{longtable}[]{@{}lll@{}}
\caption{Amostra Piloto}\tabularnewline
\toprule
Linha & Coluna & QI\tabularnewline
\midrule
\endfirsthead
\toprule
Linha & Coluna & QI\tabularnewline
\midrule
\endhead
1 & 11 & 96\tabularnewline
2 & 2 & 100\tabularnewline
2 & 13 & 89\tabularnewline
3 & 4 & 90\tabularnewline
3 & 15 & 95\tabularnewline
4 & 6 & 94\tabularnewline
4 & 17 & 87\tabularnewline
5 & 8 & 104\tabularnewline
5 & 19 & 85\tabularnewline
6 & 10 & 84\tabularnewline
\bottomrule
\end{longtable}

A amostra piloto tem como média amostral \(\overline{y}_p\) = 92.4 e
desvio padrão \(s_p\) = 6.55.

\[
n_0=\bigg(\dfrac{1.96*6.95}{0.03*92.4}\bigg)^2\cong 24
\]

\[
n = \dfrac{24}{1+\frac{24}{1000}}\cong 23
\]

Portanto, com base na amostra piloto, o tamanho da amostra deve ser 23.

\paragraph{Recalcular Parâmetros}\label{recalcular-parametros}

\begin{longtable}[]{@{}lll@{}}
\caption{Dados sorteados após cálculo do n}\tabularnewline
\toprule
Linha & Coluna & QI\tabularnewline
\midrule
\endfirsthead
\toprule
Linha & Coluna & QI\tabularnewline
\midrule
\endhead
1 & 11 & 96\tabularnewline
2 & 2 & 100\tabularnewline
2 & 13 & 89\tabularnewline
3 & 4 & 90\tabularnewline
3 & 15 & 95\tabularnewline
4 & 6 & 94\tabularnewline
4 & 17 & 87\tabularnewline
5 & 8 & 104\tabularnewline
5 & 19 & 85\tabularnewline
6 & 10 & 84\tabularnewline
9 & 10 & 89\tabularnewline
9 & 13 & 89\tabularnewline
12 & 13 & 77\tabularnewline
12 & 15 & 97\tabularnewline
15 & 15 & 91\tabularnewline
29 & 1 & 99\tabularnewline
38 & 13 & 87\tabularnewline
28 & 15 & 83\tabularnewline
14 & 20 & 100\tabularnewline
42 & 15 & 80\tabularnewline
47 & 12 & 90\tabularnewline
16 & 9 & 88\tabularnewline
34 & 15 & 88\tabularnewline
\bottomrule
\end{longtable}

Ao reaclcular, a média amostral \(\overline{y}_p\) = 90.52 e desvio
padrão \(s_p\) = 6.76.

\section{Trabalho 6}\label{trabalho-6}

\subsection{Exercício 6}\label{exercicio-6}

Considere que uma petição está sendo lançada para um determinado pedido
de uma certa causa. Na petição, ao todo, existem 676 páginas onde cada
uma delas tem um determinado número de assinaturas. Foi selecionada uma
amostra de 50 páginas registrando-se o número de assinaturas em cada uma
delas. Com base nesta amostra estime o número total de assinaturas e
determine o intervalo com 80\% de confiança.

Na tabela abaixo são apresentados os resultados obtidos na amostra

\begin{longtable}[]{@{}cc@{}}
\caption{Dados do Exercício 6}\tabularnewline
\toprule
Nr de Assinaturas & Frequência\tabularnewline
\midrule
\endfirsthead
\toprule
Nr de Assinaturas & Frequência\tabularnewline
\midrule
\endhead
42 & 23\tabularnewline
41 & 4\tabularnewline
36 & 1\tabularnewline
32 & 1\tabularnewline
29 & 1\tabularnewline
27 & 2\tabularnewline
23 & 1\tabularnewline
19 & 1\tabularnewline
16 & 2\tabularnewline
15 & 2\tabularnewline
14 & 1\tabularnewline
11 & 1\tabularnewline
10 & 1\tabularnewline
9 & 1\tabularnewline
7 & 1\tabularnewline
6 & 3\tabularnewline
5 & 2\tabularnewline
4 & 1\tabularnewline
3 & 1\tabularnewline
\bottomrule
\end{longtable}

Com média \(\overline{y}\) = 29.64 e desvio padrão \(s\) = 14.98

\subsubsection{Resolução}\label{resolucao-5}

\[
IC = 29.64\pm 2.28\sqrt{\dfrac{676-50}{676}}\dfrac{14.98}{\sqrt{50}}=29.64\pm 2.61 \rightarrow IC = (27.03;32.25) \cong (27;32)
\]

Portanto, com uma confiançã de 80\% o intervalo (27;32) contém a
verdadeira média de assinaturas por página da petição.

Logo, estima-se que o intervalo (18252;21632) contenha a quantidade
total de ssinaturas da petição.

\section{Trabalho 7}\label{trabalho-7}

\subsection{Teorema 5}\label{teorema-5}

O cálculo do IC para proporção de população finita é dado por:

\[
IC=\hat{p}\pm Z_{1-\frac{\alpha}{2}}\sqrt{\dfrac{N-n}{N}}\sqrt{\dfrac{pq}{n-1}}+\dfrac{1}{2n}
\]

\subsection{Exercício 7}\label{exercicio-7}

Em uma amostragem aleatória simples de tamanho n =100selecionada de uma
populaçãode tamanho N = 500, há 37 unidades na classe C. Ache os limites
do intervalo com 95\% de confiança para proporção e para o número total
de unidades da classe C.

\subsubsection{Resolução}\label{resolucao-6}

\[
\hat{p}=\dfrac{37}{100}=0.37 \;\;e\;\;  q=1-0.37=0.63\\IC=0.37\pm1.96\sqrt{\dfrac{500-100}{500}}\sqrt{\dfrac{0.37*0.63}{100-1}}+\dfrac{1}{2*100}=0.37\pm0.09
\]

Logo, o IC para a proporção mostra que com 95\% de confiança o intervalo
(0.28;0.46) contenha a verdadeira proporção e o intervalo (28;46)
contenha o número total real de unidades na classe C.

\section{Trabalho 8}\label{trabalho-8}

\subsection{Teorema 6}\label{teorema-6}

Para calcular a estimação de uma razão, usam-se as seguintes fórmulas:

\[
\hat{R}=\dfrac{\sum_{i=1}^nx_i}{\sum_{i=1}^ny_i}\\\;
\hat{V}[\hat{R}]=\dfrac{(N-n)}{N}\dfrac{1}{n\overline{x}^2}\bigg[\dfrac{\sum_{i=1}^ny_i^2-2\hat{R}\sum_{i=1}^ny_ix_i+\hat{R}^2\sum_{i=1}^nx_i^2}{n-1}\bigg]\\
\widehat{EP}[\hat{R}]=\sqrt{\hat{V}[\hat{R}]}\\
IC= \hat{R}\pm t_{1-\frac{\alpha}{2}}\widehat{EP}[\hat{R}]
\]

\subsection{Exercício 8}\label{exercicio-8}

Considere que uma particular região tem 1350 fazendas produtoras de
trigo. Destas fazendas deseja-se conhecer o percentual da área da
fazenda com o plantio desta cultura. Para tanto, efetuou-se um
levantamento por amostragem aleatória simples de 20 destas fazendas,
registrando em cada uma delas a área total da fazenda e a área plantada
com trigo.

\begin{longtable}[]{@{}ccc@{}}
\caption{Dados das fazendas}\tabularnewline
\toprule
Fazenda & Área Total(xi) & Área Plantada(yi)\tabularnewline
\midrule
\endfirsthead
\toprule
Fazenda & Área Total(xi) & Área Plantada(yi)\tabularnewline
\midrule
\endhead
1 & 24 & 12.5\tabularnewline
2 & 19 & 7.9\tabularnewline
3 & 27 & 11.1\tabularnewline
4 & 26 & 10.9\tabularnewline
5 & 27 & 11.2\tabularnewline
6 & 17 & 7.0\tabularnewline
7 & 15 & 7.9\tabularnewline
8 & 7 & 3.3\tabularnewline
9 & 13 & 4.1\tabularnewline
10 & 22 & 8.1\tabularnewline
11 & 8 & 3.3\tabularnewline
12 & 12 & 5.8\tabularnewline
13 & 13 & 4.4\tabularnewline
14 & 11 & 3.7\tabularnewline
15 & 5 & 2.1\tabularnewline
16 & 5 & 1.7\tabularnewline
17 & 27 & 12.3\tabularnewline
18 & 23 & 3.5\tabularnewline
19 & 31 & 12.3\tabularnewline
20 & 12 & 4.4\tabularnewline
\bottomrule
\end{longtable}

\subsubsection{Resolução}\label{resolucao-7}

\[
\hat{R}=\dfrac{137.5}{344}=0.40
\] Portanto, estima-se que 40\% da área total das fazendas são plantadas
com trigo.

\[
\hat{V}[\hat{R}]=\dfrac{(1350-20)}{1350}\dfrac{1}{2017.2^2}\bigg[\dfrac{\sum_{i=1}^ny_i^2-2*0.4\sum_{i=1}^ny_ix_i+0.4^2\sum_{i=1}^nx_i^2}{20-1}\bigg]=0.73\\
\widehat{EP}[\hat{R}]=\sqrt{0.73}=0.85
\]

O erro padrão da estimativa foi de 85\%.

\[
IC= 0.4\pm (-2.09)0.85=4\pm (-1.78)
\]

Logo, com 95\% de confiança o intervalo (38.22\%;41.78\%) contenha o
verdadeiro valor da razão da área plantada com trigo.


\end{document}
