\documentclass[]{article}
\usepackage{lmodern}
\usepackage{amssymb,amsmath}
\usepackage{ifxetex,ifluatex}
\usepackage{fixltx2e} % provides \textsubscript
\ifnum 0\ifxetex 1\fi\ifluatex 1\fi=0 % if pdftex
  \usepackage[T1]{fontenc}
  \usepackage[utf8]{inputenc}
\else % if luatex or xelatex
  \ifxetex
    \usepackage{mathspec}
  \else
    \usepackage{fontspec}
  \fi
  \defaultfontfeatures{Ligatures=TeX,Scale=MatchLowercase}
\fi
% use upquote if available, for straight quotes in verbatim environments
\IfFileExists{upquote.sty}{\usepackage{upquote}}{}
% use microtype if available
\IfFileExists{microtype.sty}{%
\usepackage{microtype}
\UseMicrotypeSet[protrusion]{basicmath} % disable protrusion for tt fonts
}{}
\usepackage[margin=1in]{geometry}
\usepackage{hyperref}
\hypersetup{unicode=true,
            pdftitle={Trabalho de Métodos de Amostragem},
            pdfauthor={Jayme Gomes dos Santos Junior},
            pdfborder={0 0 0},
            breaklinks=true}
\urlstyle{same}  % don't use monospace font for urls
\usepackage{graphicx,grffile}
\makeatletter
\def\maxwidth{\ifdim\Gin@nat@width>\linewidth\linewidth\else\Gin@nat@width\fi}
\def\maxheight{\ifdim\Gin@nat@height>\textheight\textheight\else\Gin@nat@height\fi}
\makeatother
% Scale images if necessary, so that they will not overflow the page
% margins by default, and it is still possible to overwrite the defaults
% using explicit options in \includegraphics[width, height, ...]{}
\setkeys{Gin}{width=\maxwidth,height=\maxheight,keepaspectratio}
\IfFileExists{parskip.sty}{%
\usepackage{parskip}
}{% else
\setlength{\parindent}{0pt}
\setlength{\parskip}{6pt plus 2pt minus 1pt}
}
\setlength{\emergencystretch}{3em}  % prevent overfull lines
\providecommand{\tightlist}{%
  \setlength{\itemsep}{0pt}\setlength{\parskip}{0pt}}
\setcounter{secnumdepth}{5}
% Redefines (sub)paragraphs to behave more like sections
\ifx\paragraph\undefined\else
\let\oldparagraph\paragraph
\renewcommand{\paragraph}[1]{\oldparagraph{#1}\mbox{}}
\fi
\ifx\subparagraph\undefined\else
\let\oldsubparagraph\subparagraph
\renewcommand{\subparagraph}[1]{\oldsubparagraph{#1}\mbox{}}
\fi

%%% Use protect on footnotes to avoid problems with footnotes in titles
\let\rmarkdownfootnote\footnote%
\def\footnote{\protect\rmarkdownfootnote}

%%% Change title format to be more compact
\usepackage{titling}

% Create subtitle command for use in maketitle
\newcommand{\subtitle}[1]{
  \posttitle{
    \begin{center}\large#1\end{center}
    }
}

\setlength{\droptitle}{-2em}

  \title{Trabalho de Métodos de Amostragem}
    \pretitle{\vspace{\droptitle}\centering\huge}
  \posttitle{\par}
    \author{Jayme Gomes dos Santos Junior}
    \preauthor{\centering\large\emph}
  \postauthor{\par}
      \predate{\centering\large\emph}
  \postdate{\par}
    \date{Curitiba 2018}

\usepackage{booktabs}
\usepackage{longtable}
\usepackage{array}
\usepackage{multirow}
\usepackage[table]{xcolor}
\usepackage{wrapfig}
\usepackage{float}
\usepackage{colortbl}
\usepackage{pdflscape}
\usepackage{tabu}
\usepackage{threeparttable}
\usepackage{threeparttablex}
\usepackage[normalem]{ulem}
\usepackage{makecell}

\usepackage{float} \usepackage[brazil]{babel} \usepackage{amsmath} \usepackage{bm} \usepackage{graphicx}

\begin{document}
\maketitle

{
\setcounter{tocdepth}{2}
\tableofcontents
}
\newpage

\section{Trabalho 1}\label{trabalho-1}

\subsection{Teorema 1}\label{teorema-1}

Tendo uma população de tamanho \textbf{N}, ao extrair uma amostra de
tamanho \textbf{n}, temos que o estimador não evezado para
\(\overline{Y}\) é: \[
\overline{y} = \sum_{i=1}^n\dfrac{y_i}{n}
\]

\subsection{Exercício}\label{exercicio}

\begin{table}

\caption{\label{tab:unnamed-chunk-1}População de Tamanho N = 6}
\centering
\begin{tabular}[t]{clc|clc}
\hline
População & Valores\\
\hline
y1 & 2\\
\hline
y2 & 4\\
\hline
y3 & 5\\
\hline
y4 & 7\\
\hline
y5 & 8\\
\hline
y6 & 9\\
\hline
\end{tabular}
\end{table}

\newpage

Com base na população dada na tabela 1, encontrar a média de todas as
combinações possíveis de amostras de tamanhos: n = 2, n = 3 e n = 4.

\subsubsection{Resolução}\label{resolucao}

Sabendo que a média populacional é dada por:

\[
\overline{Y} = \sum_{i=1}^N\dfrac{y_i}{N} = \dfrac{2+4+5+7+8+9}{6} = 5.833
\] O objetivo é mostrar que a média de todas as combinações de amostras
possíveis terão o mesmo valor da média populacional.

\emph{OBS:} ys = médias amostrais, P(ys) = probabilidade de cada amostra
acontecer e ys.P(ys) = média amostral multiplicada pela sua
probabilidade.

\newpage

\paragraph{Para n = 2, temos:}\label{para-n-2-temos}

\begin{table}

\caption{\label{tab:unnamed-chunk-1}Amostras Possíveis com n = 2}
\centering
\begin{tabular}[t]{clc|clc|clc|clc}
\hline
Amostra & ys & P(ys) & ys.P(ys)\\
\hline
2,4 & 3 & 1/15 & 0.2\\
\hline
2,5 & 3.5 & 1/15 & 0.23\\
\hline
2,7 & 4.5 & 1/15 & 0.3\\
\hline
2,8 & 5 & 1/15 & 0.33\\
\hline
2,9 & 5.5 & 1/15 & 0.37\\
\hline
4,5 & 4.5 & 1/15 & 0.3\\
\hline
4,7 & 5.5 & 1/15 & 0.37\\
\hline
4,8 & 6 & 1/15 & 0.4\\
\hline
4,9 & 6.5 & 1/15 & 0.43\\
\hline
5,7 & 6 & 1/15 & 0.4\\
\hline
5,8 & 6.5 & 1/15 & 0.43\\
\hline
5,9 & 7 & 1/15 & 0.47\\
\hline
7,8 & 7.5 & 1/15 & 0.5\\
\hline
7,9 & 8 & 1/15 & 0.53\\
\hline
8,9 & 8.5 & 1/15 & 0.57\\
\hline
\end{tabular}
\end{table}

Comparando os resultados das médias populacional e amostral com n = 2,
temos que:

Média populacional 5.83 = Média de todas as amostras possíveis de
tamanho 2 5.83


\end{document}
