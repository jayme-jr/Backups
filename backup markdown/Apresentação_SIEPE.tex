\documentclass[ignorenonframetext,]{beamer}
\setbeamertemplate{caption}[numbered]
\setbeamertemplate{caption label separator}{: }
\setbeamercolor{caption name}{fg=normal text.fg}
\beamertemplatenavigationsymbolsempty
\usepackage{lmodern}
\usepackage{amssymb,amsmath}
\usepackage{ifxetex,ifluatex}
\usepackage{fixltx2e} % provides \textsubscript
\ifnum 0\ifxetex 1\fi\ifluatex 1\fi=0 % if pdftex
  \usepackage[T1]{fontenc}
  \usepackage[utf8]{inputenc}
\else % if luatex or xelatex
  \ifxetex
    \usepackage{mathspec}
  \else
    \usepackage{fontspec}
  \fi
  \defaultfontfeatures{Ligatures=TeX,Scale=MatchLowercase}
\fi
\usetheme[]{Wolwerine}
% use upquote if available, for straight quotes in verbatim environments
\IfFileExists{upquote.sty}{\usepackage{upquote}}{}
% use microtype if available
\IfFileExists{microtype.sty}{%
\usepackage{microtype}
\UseMicrotypeSet[protrusion]{basicmath} % disable protrusion for tt fonts
}{}
\newif\ifbibliography
\hypersetup{
            pdftitle={A problemática dos acidentes de trânsito na Região Metropolitana de Curitiba utilizando a estatística espacial},
            pdfauthor={Marcelo José Martins Júnior},
            pdfborder={0 0 0},
            breaklinks=true}
\urlstyle{same}  % don't use monospace font for urls

% Prevent slide breaks in the middle of a paragraph:
\widowpenalties 1 10000
\raggedbottom

\AtBeginPart{
  \let\insertpartnumber\relax
  \let\partname\relax
  \frame{\partpage}
}
\AtBeginSection{
  \ifbibliography
  \else
    \let\insertsectionnumber\relax
    \let\sectionname\relax
    \frame{\sectionpage}
  \fi
}
\AtBeginSubsection{
  \let\insertsubsectionnumber\relax
  \let\subsectionname\relax
  \frame{\subsectionpage}
}

\setlength{\parindent}{0pt}
\setlength{\parskip}{6pt plus 2pt minus 1pt}
\setlength{\emergencystretch}{3em}  % prevent overfull lines
\providecommand{\tightlist}{%
  \setlength{\itemsep}{0pt}\setlength{\parskip}{0pt}}
\setcounter{secnumdepth}{0}

\title{A problemática dos acidentes de trânsito na Região Metropolitana de
Curitiba utilizando a estatística espacial}
\author{Marcelo José Martins Júnior}
\date{PET-Estatística}

\begin{document}
\frame{\titlepage}

\begin{frame}{Objetivos}

\begin{itemize}[<+->]
\item
  Objetivo geral: Indicar qual setor relacionado ao trânsito de Curitiba
  e Região Metropolitana (RMC), possui maior concentração de acidentes
  de trânsito.
\item
  Objetivos específicos:
\end{itemize}

\begin{enumerate}[<+->]
\def\labelenumi{\arabic{enumi}.}
\tightlist
\item
  Indicar o setor de maior problemática geoespacial na concentração de
  acidentes;
\item
  Com o uso da estatística espacial, bem como, informações de órgãos de
  monitoramento do trânsito de Curitiba e RMC, estudar as regiões e
  setores deficientes na questão de concentração de acidentes;
\item
  Comparar e analisar os dados da pesquisa de estatística espacial, e os
  dados existentes fornecidos por grandes órgãos do estado.
\end{enumerate}

\end{frame}

\begin{frame}{O que é a Estatística Espacial?}

\begin{itemize}[<+->]
\item
  Conjunto de métodos de análise de dados em que a localização
  geográfica é usada explicitamente na modelagem
\item
  Vale ressaltar que nem todos os dados fazem uso da estatística
  espacial na análise.
\end{itemize}

\end{frame}

\end{document}
